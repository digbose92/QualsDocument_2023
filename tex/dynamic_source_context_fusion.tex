\chapter{Multimodal context-based large-scale semantic video understanding}
\section{Transition from static scenes to dynamic content}
\section{Semantic video understanding}
The need for understanding video semantics beyond actions and objects 

\subsection{Contextual signals as multi-modal streams:}
    \subsubsection{Visual scene context}
    \subsubsection{Auditory context}
    Background music + audio events + speech (vocalizations)
    \subsubsection{Language based context}
\section{Advertisements as medium}
\subsection{Challenges}
Mention three cases:
\begin{itemize}
\item Visual context changes sharply across shots but is tied together by an overall narrative (a diagram especially)
\item The visual narrative doesn’t happen in isolation and is usually accompanied by a transition in musical tone (the contextual signal present in the background score/music sets the mood for the advertisement)
\item The verbal cues provide descriptions of the foreground activities, including character interactions and narrations about the overall theme/topic (Language driven foreground context representations)
\section{MM-AU dataset}
\subsection{Data sources}
\subsection{Human expert-driven annotations}
\subsection{Dataset statistics}
\section{Multimodal representative tasks}
\subsection{Topic categorization}
\subsection{Tone transition}
\subsection{Absence/Presence of socially relevant cues}
\section{Progressive multimodal fusion of contextual streams}
\section{Experiments and Results}
\subsection{Language-only reasoning}
\subsection{Unimodal vs Multimodal baselines}
\section{Limitations}
\section{Conclusions}
\end{itemize}




